\rhead{15/12/2015}
\section{La protection du secret défense (IGI 1300)}
\subsection{Principe et organisation}
Quand on parle de classification Défense, on parle des classifications de l'Etat et non du Ministère de la Défense seulement. Pour le secret, on va retrouver les classifications suivantes, du plus haut niveau au plus bas niveau :
\begin{itemize}
 \item Très Secret Défense
 \item Secret Défense
 \item Confidentiel Défense
 \item[~] \textbf{\textcolor{red}{-----------------------------------------------------------}}
 \item Confidentiel Spécifique
 \begin{itemize}
  \item Confidentiel Professionnel
  \item Confidentiel Financier
  \item Confidentiel Médical
  \item Confidentiel Concours (Baccalauréat)
 \end{itemize}
 \item Diffusion Restreinte
 \item Non Protégé
\end{itemize}
Les trois classifications les plus hautes sont régies par la loi Française. Dans les autres cas, ce sont les règlement intérieurs du corps de métier qui traitent les sanctions.\\
Ainsi, dans le cas de la loi, on parle de sanction correctionnelle (prison et amende). Pour les règlements intérieurs, on peut se retrouver licencié pour faute professionnel, faire face à des sanctions disciplinaires.\\
Tout ce qui est au dessus du NP est appelé \enquote{\textit{sensible}}. Les DR et CS sont les \enquote{\textit{Sensible non classifié de Défense}}. Enfin, les trois autres sont \enquote{\textit{Sensible classifié de Défense}}.\\
L'IGI 1300 ne traite que les documents et informations sensible classifié de Défense.\\
Une des particularités du TSD est qu'il ne peut pas être traité de manière numérique.\\
Afin de définir une sanction, le juge va chercher à savoir si l'action est volontaire ou non. En cas d'action volontaire, la peine sera entre la maximum et la moitié de la peine. En cas d'action non volontaire, la peine sera entre le minimum et la moitié. Ensuite, on cherche le coup monétaire et le coup moral.\\~\\\par
Lorsque l'on parle de classifié de Défense, on parle de la politique, du milieu militaire, diplomatique, scientifique, etc\ldots\\
Pour les niveaux de classification, on va choisir en fonction du niveau de nuisance à l'atteinte de l’État entre TSD (nuire très gravement), SD (nuire gravement) ou CD (nuire).
\subsection{Mesures de sécurité liées aux personnes}
Afin qu'une personne puisse avoir accès à un certain type de documents classifiés, cette dernière doit être habilité à ce niveau de classification. On retrouve aussi le \textit{besoin d'en connaître}. Ce dernier ne permet à un personne de n'avoir accès qu'au documents dont vous avez besoin.\\
Afin de recevoir une habilitation, il faut au préalable remplir une notice 94-A. La notice est alors traitée par la DGSI ou la DPSD (pour le Ministère de la Défense). Ces deux agences ne donnent alors qu'un avis quand à la personne. Les avis ont une validité de 5 ans pour le TSD, 7 ans pour le SD et 10 pour le CD. C'est ainsi l'autorité qui a demandé l'avis qui choisi s'il souhaite suivre l'avis ou non.
\subsection{Mesures de sécurité liées aux informations, aux supports}
L'objectif est de mettre sur un support un visuel permettant à ce que les personnes sachent si elles peuvent accéder à ce document. Pour cela, on va mettre des tampon en haut et en bas de la feuille comprenant le niveau de classification. Sur la première feuille, on va aussi préciser combien de temps ce document va être classifié ou dans combien de temps le document sera déclassifié (in descend d'un cran au niveau de la classification).
\subsection{La protection des lieux}
On va retrouver différents niveau de protection des batiments : 
\begin{description}
 \item[Classe 4] 
 \item[Classe 3] Classe 4 + \ldots
 \item[Classe 2] Classe 3 + \ldots
 \item[Classe 1] Classe 2 + \ldots
\end{description}
Pour les locaux, on retrouve la classification suivante :
\begin{description}
 \item[Classe d] 
 \item[Classe c] Classe d + \ldots
 \item[Classe b] Classe c + \ldots
 \item[Classe a] Classe b + \ldots
\end{description}
Pour la partie local, on trouve aussi les serrures. Ces dernières ne sont pas normées.\\
Enfin, pour le meuble, nous pouvons retrouver les classes suivantes :
\begin{description}
 \item[Classe C] 
 \item[Classe B] Classe C + \ldots
 \item[Classe A] Classe B + \ldots
\end{description}
Les zones comprenant des informations importantes sont des \textit{zones protégées}.
\subsection{Protection des systèmes d'information}
Il y a des contrôles de sécurité (ici plus des contrôles que des audits). Pour sécuriser un système d'information il y une sécurité matérielle, d'accès au matériel. Dans le milieu militaire, il y a un poste pour chaque réseau : un pour le NP, un pour l'intradef (DR et CS) puis un pour chaque niveau de classification.\\
Lorsque l'on va vouloir relier deux réseau de niveaux de classification différents, il va falloir utiliser des appliances agrées par l'ANSSI afin de communiquer (par exemple, des VPN CD/SD qui passent par des VPN du niveau inférieur jusqu'au NP).
\subsection{Protection du secret dans les contrats}
\subsection{Rapport 2015 + Film FIC + ARGO}
Dans le début du film Argo, il faut évacuer l’ambassade. Comment cela se passe ?