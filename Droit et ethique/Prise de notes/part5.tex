\section{Sécurité des Activités d'Importance Vitale (SAIV)}
Il y a entre 100 et 200 entreprises qui sont OIV en France, dans 7 secteurs. Elles doivent élaborer des plans de sécurité dans lesquels ils mettent en place des lieux stratégiques (sites d'importance vitale) qu'ils doivent protéger selon des normes de l’État.
\section{La protection du potentiel scientifique et technique}
Le but est de préserver l'innovation scientifique et technique. Cela ne s'applique pas qu'à la recherche, au contraire. Toutes les entreprises qui ont une valeur à protéger, elles vont pouvoir appliquer les normes des PPST. Ce qui est à protégé est une ZRR\footnote{Zone R... R...}. En cas de problème, le nouveau dispositif est conçut pour que l'Etat puisse aider ces ZRR (par exemple, en cas de risque, une voiture de police passera devant la ZRR tous les jours).\\
Contrairement aux locaux abritant des informations sensibles confidentielles Défense, la ZRR peut être seulement un pièce, un meuble, sans que le bâtiment complet soit impacté.
\section{Projet}
Monter un entreprise (nom, logo). On a une emprise géographique en France. On a une entreprise de X personnes. On vend qqch avec une valeur que l'on veut protéger.\\
Qui ?\\
Où ?\\
Combien ?\\
Qui bosse où ?\\
Comment monte-t-on le dossier ZRR ?