\rhead{14/12/2015}
\section{Introduction}
Ici, nous allons parler de l'IGI900\footnote{Instruction Générale Interministérielle}. Ce texte a pour but de définir les normes de sécurité des SSI.\\
Nous pouvons aussi étudier la Recommandation 600 qui traite des données confidentielles.\\
On va retrouver des QPC\footnote{Question Prioritaire de Conformité}.\\
Les arrêtés n'ont pas le droit d'élargir une loi, juste de la restreindre (exemple : la limitation de vitesse est de 50km/h en vile (loi). Un ville peu faire un arrêté municipale pour appliquer la limitation à 30 ou 40km/h).\\
Le dernier type de droit est un contrat.\\
A part la loi fiscale, aucune loi n'est rétro-active.\\
Dans le milieu militaire, un RSSI est appelé OSSI\footnote{Officier de la Sécurité des Systèmes d'Information}.\\
Il est de plus en plus difficile de définir ce qu'est un système d'Information (linki met en place une CPL outdoor).\\
La sécurité des systèmes d'information est réalisée en trois parties : la confidentialité de l'information, la disponibilité et l'intégrité.
\section{Gestion du risque}
\subsection{Métrique du risque}
L'idée est d'avoir des mesures relatives du risque. Le but est d'assurer un soutien à la direction pour lui permette de réaliser un choix.\\
Le risque peut se caractériser comme étant la probabilité qu'une menace particulière puisse exploiter une vulnérabilité donnée du système. Le risque est, souvent, une multiplication de la menace et de la vulnérabilité.\\
Il existe plusieurs méthodes d'analyse de risque. Ici, nous verrons la méthode EBIOS\footnote{Expression des Besoins et Identification des Objectifs de Sécurité}.
\section{L'organisation de la SSI au niveau interministériel}
L'organisation interministérielle est à l'heure actuelle maillée du haut niveau au bas niveau (du niveau interministériel au niveau ministériel). Les différents responsables du SSI ont tous les mêmes rôles.
\subsection{Contrôle du personnel}
Il faut tenir à jour une liste du personnel ayant accès à des données restreintes.
\subsection{Protection de l'information}
L'IGI1300 définie les protections étatiques (très secret défense, secret défense, confidentiel défense). Il existe des classification de plus haut niveau :
\begin{itemize}
 \item Secret FOCAL (Europe)
 \item Secret OTAN
 \item Secret politique (Saphir, Diamant)
\end{itemize}

On va retrouver d'autres types de protections comme le secret médical.\\
Pour une bonne gouvernance de l'entreprise, il ne faut pas que le RSSI rende compte au DSI mais directement au DG afin d'être certain qu'aucune information ne soit filtrée.