\section{Loi de programmation militaire}
\subsection{Organismes d'Importance Vitale}
Les OIV sont partout. Que ça soit dans les domaines étatiques ou l'agriculture, l'énergie, les finances, l'écologie, la santé, l'audiovisuel et l'information, les communications, etc\ldots
\section{Les aspects juridique de la sécurité des systèmes d'information}
La CNIL est composée de 17 personnes élue pour 5 ans.\\
Il y a trois organisations qui échappent au contrôle de la CNIL : Les organisation syndicales, les organisation philosophiques et religieuses et les journalistes.
\section{La protection des bien immatériels}
La loi du 1 juillet 1992 est le texte fondateur sur la propriété intellectuelle. La loi du 5 février 1994 est pour la répression de la copie de la propriété intellectuelle.
\subsection{Les droits d'auteur}
Dans le cas d'une réalisation personnelle dans un environnement personnel, le créateur est le titulaire des droits d'auteur.\\
Dans le cas d'une réalisation dans le cadre professionnel, par un employé, c'est l'entreprise qui est titulaire des droits d'auteur.\\
Dans le cas d'un logiciel, cela protège tout, de la préparation à l'exécutable. Le créateur, la personne physique ou morale est protégé.\\
Quand une personne a le droit d'utilisation d'un logiciel, il est possible de faire une copie de sauvegarde du logiciel.
\section{La protection des systèmes de données}
Un eMail est une correspondance écrite. L'interception des communications est interdite. L'informaticien n'est pas soumis au secret professionnel mais à la discrétion.\\
La loi Godfrain est maintenant la loi 323 du code pénal.\\
On parle de criminalité informatique lorsque un outil informatique est utilisé en temps que moyen ou outil de réalisation de fraude.\\
Si un individu détruit des données, il n'est pas possible d'utiliser ces données contre lui.
\section{L'action judiciaire des partenaires de la SSI}
Il faut poser déposer une plainte en cas de fraude informatique avec ou sans préjudice.\\
On ne peut pas parler de crime ou de démit s'il n'y a pas d'intention maligne. On peut cependant parler de délit s'il y a un manquement à une obligation de prudence, sécurité.