\rhead{16/11/2015}
\section{Introduction}
Tous les états ont la copie des certificats des grands groupes (Google, Microsoft, etc\ldots).\\
La cryptologie c'est la science du secret. Il faut savoir contre quoi et qui on se protège, et comment.\\
Un système d'information est un système fonctionnant à l'électricité, traitant de l'information depuis sa création jusqu'à sa destruction.\\
Un système d'information peut être soumis à deux types d'attaques :
\begin{itemize}
 \item Les actions naturelles, non malveillantes, non ciblés. On parle alors de sûreté.
 \item La sécurité. On a ici un caractère ciblé, adaptatif, imprévisible.
\end{itemize}
Ici, on ne s'occupera que de la sécurité.On va parler de cryptographie dans la défense et de cryptanalyse dans l'attaque.\\
Pour contrer les attaques, il va falloir protéger l'information. Pour cela, on va passer d'un réseau rouge (non chiffré) à un réseau noir (chiffré). On va ainsi parler de COMSEC (Communication Security). Ici le canal n'est pas sécurisé. Pour sécuriser aussi le canal, on va parler de TRANSEC\footnote{Transmission Security} comme avec la stéganographie.
\subsection{Les acteurs de la cryptologie}
Il y a un ou plusieurs émetteurs, un ou plusieurs destinataires. Cela se fait sur un canal ou un réseau public. Il peut y avoir un ou plusieurs intrus malveillants. Enfin, il peut y avoir un ou plusieurs messages.
\section{Les menaces et solutions de la cryptologie}
La cryptologie est partout.\\
On va pouvoir retrouver quatre menaces :
\begin{itemize}
 \item Atteinte à la confidentialité des données. E va écouter A.
 \item E envoie un message à B en se faisant passer par A. On a lors un problème d'identification et d'authentification.
 \item E modifie un document. On a alors un problème d'intégrité.
 \item A envoie un message à B puis le répudie. On a alors un problème de signature.
\end{itemize}
Il y a aussi des problèmes annexes tels que :
\begin{itemize}
 \item Les problèmes d'accusés de réception; d'émission
 \item Les problèmes de falsification de la date de création d'un message, d'un problème d'horodatage.
\end{itemize}
Pour ces deux problèmes, on peut mettre en place des systèmes de vérification poussée. Cependant, au bout d'un moment, on doit se baser sur la confiance entre humains.\\
On a aussi les problèmes TRANSEC : brouillage de communication, coupure de canal, A qui veut cacher l'envoie d'un message.
\subsection{Les solutions}
\begin{description}
 \item[La confidentialité] Les données doivent rester initelligibles à toutes personnes non destinataires
 \item[L'intégrité] 
 \item[L'authentification] 
 \item[La signature] 
\end{description}
\subsection{Le chiffrement}
\subsubsection{Chiffrement symétrique}
A et B ont la même clé secrète pour chiffrer et déchiffrer.\\
Le problème reside dans la gestion de clés car A et B soivent avoir les clés au préalable de la communication.
\paragraph{Chiffrement par flot}~\\\par
Système qui opère individuellement sur chaque bit du clair en utilisant un transformation qui varie en fonction du bit d'entrée. C'est un système de chiffrement à la volé.\\
On va beaucoup retrouver ce système de chiffrement dans les systèmes de communications.\\\par
L'avantage de ce chiffrement est la rapidité. Son implémentation en terme de porte logique est beaucoup plus simple en électronique. Cela peut permettre un système de secret parfait.\\
Le gros inconvéniant réside dans la mise en place et la distribution de clés. Le secret parfait est aussi difficile à réaliser (clé assez longue pour être de la même taille que le clair).
\subsubsection{Chiffrement asymétrique}
On va ici couper l'information (qui est longue) en blocs.\\
Système qui divise le clair en blocs de taille fixe et chiffre un bloc à la fois. La taille des blocs est en général de 64 à 128 bits.\\
Dans un système de chiffrement à clés publiques, chaque utilisateur dispose d'un couple de clés : une clé privée et une clé publique.\\
La clé privée n'est connue que de l'utilisateur qui l'a créée.\\
D'un point de vue mathématique, on utilise les fonctions mathématiques à sens unique avec trappe.
\paragraph{Fonction à sens unique}~\\\par
Une fonction $f:M\rightarrow C$ est une FSU si :
\begin{itemize}
 \item Connaissant $m\in M$, il est facile de calculer $f(m)$
 \item Connaissant $f(m)$, il est difficile de trouver $m$
\end{itemize}
\paragraph{Fonctions avec trappe}~\\\par
Une telle fonction $f$ est dite avec trappe, si de plus la connaissance de $f(m)$ est d'une information supplémentaire (clé privée) permet de retrouver $m$ facilement.\\
La construction repose de ces fonctions repose sur des problèmes mathématiques dits très difficile : factorisation, résidu quadratique, logarithme discret, problèmes d’empilement, etc\ldots\\
La sécurité de ces systèmes repose sur la supposition (non démontrée) qu'il existe des problèmes véritablement difficiles.\\\par
L'avantage de ce chiffrement, il n'y a pas d'échange de clé nécessaire.\\
Cependant, ce système est très lent et n'est que supposé sûr\footnote{Snowden dual Random Bit Generator}.
\subsubsection{Les certificats}

\subsubsection{Chiffrement hybride}
On chiffre le message de manière symétrique. On chiffre la clé de manière asymétrique. On envoie les deux chiffrés (la clé chiffré en asymétrique et le message en symétrique).