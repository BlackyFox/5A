\rhead{07/10/2015}
Pour administrer notre pare-feu, on peut passer par le SSH. Pour cela, on doit :
\begin{itemize}
 \item Installer le service sshd serveur
 \item Le configurer sur un port de notre choix (ici, port 1313)
 \item On va ensuite autoriser le SSH à passer sur le firewall
\end{itemize}
Pour visualiser les log, il faut taper (sous Mageia) journalctl -u ssh -f
\section{NAT}
Le SNAT c'est quand on change l'IP source. Les adresses IP du réseau local sont alors remplacées par l'adresse du routeur.\\
Sur le client, on marque :
\begin{lstlisting}[style=custombash]
iptables -t NAT -A POSROUTING -o externe -j SNAT --source
\end{lstlisting}
A partir du moment que ça sort par la carte externe, on fait le NAT.
\rhead{08/10/2015}
L'option MASQUERADE permet de faire du SNAT quand on n'a pas d'IP fixe dans le réseau local. Ainsi, on n'a pas à mettre des règles pour les clients.