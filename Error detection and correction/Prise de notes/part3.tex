\rhead{22/09/2015}
\section{Linear error-correcting codes}
\subsection{Introduction}
On va structurer les codes.\\
Un code linéaire est n'importe quel EV d'un EV général.\\
On note ainsi un code \[[n, k, d]\]
Avec $k$ la dimension de l'espace.\\
Dimension $k$ = base fait $k$.\\
Un code linéaire $[n k, d]$ peut s'écrire comme code général de la façon suivante : $(n,q^k,d)$.\\
K vecteur de taille n (k lignes, n colonnes.\\
Poids de Haming = nombre de données non nulles dans un code binaire.\\
La matrice de parité peut être plus petite qua la matrice génératrice.
\subsection{Using linear codes (encoding)}
\begin{equation}
 \Sigma^k_{i=1}a_ir_ik
\end{equation}
\subsection{The minimum distance of linear codes}
Coset = translaté
\subsection{Minimum-distance Decoding for Linear Codes - Syndrome}
Coset leader = coset au poids minimal
\subsection{Application}

\subsection{Conclusion}
