\rhead{09/11/2015}
\section{Introduction incident investigation}
\subsection{Fundamentals}
\subsubsection{EDRM}
EDRM\footnote{Electronic discovery reference model} is about finding a international model for investigations.\\
There are 9 stages in the all model :
\begin{enumerate}
 \item Information management (T0)
 \item Identification (T1)
 \item Preservation (T2)
 \item Collection (T2)
 \item Processing (T3)
 \item Review (T3)
 \item Analysis (T3)
 \item Production (T4)
 \item Presentation (T5)
\end{enumerate}
Sometimes, you can find some pieces of material that forces you to go back on the model. This is why there is some arrows going down and back.\\
The fourth part is the volume (yellow). It express the need of reducing the size of the volume (huge at the beginning, small at the end).\\
The fifth element is the relevance of the data. At the end you must have few data but relevant data.\\
The beginning of the investigation doesn't start at the first stage. This first stage is about knowing where are the files, what system is running, etc\ldots They start when \enquote{\textit{shit appends}}.\\
This often start when there is litigation (suing). In the USA, when someone says \enquote{I'm going to sue you}, the target have the obligation to freeze all his data. This is called litigation hold.\\
In Europe, I can't have the right to demand some data to the other to prove the case (like in the USA). In Europe, you have to find your proofs all by yourself and present them to a judge.
\paragraph{Information management (T0)}~\\
To manage all sources of Information, including ESI. It does that by defining and implementing for all sources of Information, especially for ESI.\\
Data can be in rest, idle, in the cloud, out of used (archived, tapes, special hard disks, etc\ldots) but you legally obliged to keep them for legal obligations.\\
Then, you need some retention policies : how the data will be stored, what data can be destroy (then you need a legal department to know what data you can destroy. If you destroy wrong data, you can be sued), how long the data should be stored\ldots\\
You can also have some e-Discovery-processes ready when something wrong happen.\\
But most of the time, it is not so great, is most of the time a mess for every process. This is a lot about documentation. This part is often skipped because of lack of time or laziness.\\
\paragraph{Identification (T1)}~\\
This stage is about determining what should be preserved and collected, the data that should be kept. You should also determine the scope, the breadth and depth of needed ESI.\\
So you're simply making a list of what you want.
\paragraph{Preserve}~\\
\paragraph{Collect}~\\
In this part you want to get the information you have identified. You want this data exactly the same.
\paragraph{Process}~\\
You need to extract the data, convert it to make it more readable, remove non-relevant data, scan the papers to make some faster searches on it, get rid of the duplicates, remove data out of the time scope you selected, etc\ldots
\paragraph{Review (T5)}~\\
Then, you need some tools to get through the data really faster. This is not only about finding some specific words, it can be to check the relevancy of the data too.\\
You also have to get rid of the privileged information (lawyer - client, doctor - client, priest - believer, etc\ldots). You can't look at those information as an investigator or you will surely loose the case.\\
The data given to the judge are only non-privileged ones.
\paragraph{Analysis (T6)}~\\
Here, you are trying to build the picture of what happened.\\
To do so, you need to build some relation diagrams, some activities lists, etc\ldots\\
You also need to determine some specific vocabulary, specific keywords that are signals for the scheme of malicious people.\\
This part is more about mind work, putting all the pieces together.
\paragraph{Production (T7)}~\\
Now that you know what happened, how did it happened, you can produce some responsive documents for the client. You need to know how the client wants the information (reports, excel spreadsheet, pdf, raw data, etc\ldots).
\paragraph{Presentation (T8)}~\\
Then, you also need to make a presentation. This can be only for the client or as an expert witness in court (in front of the jury).
\subsection{The trigger}
A trigger can be an unexpected event that differs from the normal business operation and has caused/causes/might cause harm or damages.\\
The incident is something from the past. The signal/warning is the present. And the uncertainty is for the future.\\
If there is a trigger, there is an assignment. This assignment differs between the different triggers.
\begin{table}[H]
 \centering
 \begin{tabular}{|l|c|c|}
 \hline
  Incident & Past & Reconstruction\\\hline
  Signal/Warning & Present & Audit/Inspection\\\hline
  Uncertainty & Future & Exploration\\\hline
 \end{tabular}
 \caption{Assignments for specific triggers}
\end{table}
Depending on the type of assignment, you have to answer to different questions :
\begin{table}[H]
 \centering
 \begin{tabular}{|l|c|}
 \hline
  Reconstruction & What happened? How did it happened?\\\hline
  Signal/Warning & What is happening? How is it happening?\\\hline
  Uncertainty & What might happen? How might this happen?\\\hline
 \end{tabular}
 \caption{Question for specific assignments}
\end{table}
\subsubsection{The investigation : ...}
For an investigation, you have to be able to ask 8 \enquote{w} questions :
\begin{table}[H]
 \centering
 \begin{tabular}{|l|c|}
 \hline
  What & Intelligent data Analysis\\\hline
  Who & Business investigator\\\hline
  Where & Location\\\hline
  When & Time period\\\hline
  With what & Means\\\hline
  Which way & Approach\\\hline
  Why & (Re)Construction\\\hline
  What for & Story\\\hline
 \end{tabular}
 \caption{Eight question you must ask during an investigation}
\end{table}
\subsubsection{The investigation: the model}
You can use some tools to make some mind mapping such as the one after.\\
An investigator doesn't have any authority regarding the data he can use. He also have to be independent and impartial. Even if the result doesn't please the client.\\
The first thing you have to do when you have an assignment is asking questions about the goal of the client, about the data that was given to us, or even about the assignment itself.
\subsubsection{The investigation: Digging for\ldots}
An investigator is here to find some facts, not truth or anything else.\\
\subsection{The job}
\subsubsection{investigation is like archeology}
There are 5 phases :
\begin{enumerate}
 \item Preliminary research: Where to dig?
 \item Field work: The digging and the findings
 \item Lab work: Process the findings
 \item Desk work: (Re)Construction the issue
 \item Closing: Present the results
\end{enumerate}
The investigator find some things like artefact's and traces. Since those things don't talk, it is the investigator duty to tell their story. He has to reconstruct the past and present or construct the future.
\subsection{The process}
\subsubsection{Phase 1 : Preliminary researches}
The first case analysis is about searching for background information on the case using given documents and public sources about all persons, places, times involved.\\
Here, we're giving a scope. We don't have to get out of this scope for the case.\\
\enquote{I found very likely that\ldots}