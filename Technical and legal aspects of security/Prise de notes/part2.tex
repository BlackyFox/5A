\section{Incident response: Legal and documentation aspects}
\subsection{Important stuff}
e-Discovery is not about technical stuff, more on static forensic, live forensic, order of volatility, etc\ldots\\
It's always better to write things down before making any action. It helps organizing your mind and stay legal.\\
No matter how sure you are about a legal thing, you must always call your lawyer first.
\subsection{Civil law}
Basicaly, it's about \enquote{A against B for some reason}. What is asked is a compensation (usually, money). Both of the parties collect their proofs (documents (any kind of documents, papers, digital records, etc\ldots)). Then, a court of law decide. Civil law is a balance of probabilities.\\
To get some stuff from the other party, you can't get it by betting on the will of B. To get the evidence, you have the Anton Piller Order and the Norwich Pharmacal Order. Those are both court orders. So you have to convince a judge to give one to you.
\subsubsection{Anton Piller Order}
Search an seizure without prior warning. 
\subsubsection{Norwich Pharmacal Order}
A and B goes at the court and a third party to get A's requirement against B. In France, C, the third party, is Hadopi. This order is under the European Convention of Human Rights - Article 8 \cite{ECHR}. Since this is a privacy violation, you must have the court authorization.
\subsection{Criminal laws}
Here, it's society (people, states, etc\ldots) against \enquote{bad guys}. It is about murder, theft (financial, identity, shoplifting, etc\ldots).\\
Since this is the state and not M. A, they have other rights:
\begin{itemize}
 \item They can arrest suspects
 \item They can collect some evidence (interogation, search and seizure, etc\ldots) usually overseen by the police
\end{itemize}
~\\\par
The difference between Civil law and Criminal law is that criminal law is about punishing people and not about compensation.\\
~\\\par
Since we can't know at the beginning if this is going to be a criminal or an civil case, we always assume that it's a criminal one. Then, to be able to prove it's really a criminal case (to involved the police force), you have to document everything you do.
\subsection{Documentation aspects}
You can read the ACPO\footnote{Association Chief Police Officer} guideline to get some ideas about taking some data:
\begin{itemize}
 \item Every time you can, copy the data, don't work on the originals.
 \item ...
 \item Don't be an idiot
 \item The person in charged of this investigation is responsible that the law and these principles are respected.
\end{itemize}