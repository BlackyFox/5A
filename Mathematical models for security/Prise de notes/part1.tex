\rhead{12/10/2015}
\section{Théorie des graphes}
\subsection{Définitions et généralités}
\begin{description}
 \item[Les \textit{p-graphes} (orientés)] C'est un couple $G (X,U)$. $X$ est l'ensemble des nœuds et $U$ l'ensemble des arcs. $U$ est une famille et non un ensemble. Ainsi on peut avoir plusieurs fois un élément représenté : $U={u_1, u_2, u_2, u_3, \ldots}$.\\GRAPHE\\Pondérer un graphe c'est faire une application C pour laquelle $C:U\rightarrow R$.\\
 Boucle $u\in U | (x_i, x_i)$ 
 \begin{itemize}
  \item $w^+(x)$ c'est l'ensemble des arcs qui arrivent sur le nœud
  \item $w^-(x)$ c'est l'ensemble des arcs qui partent du nœud
  \item $w(x)=w^{+}(x)\bigcup w^{-}(x)$
  \item $|X|=N$
  \item $|U|=M$
 \end{itemize}
 \item[1-graphes] Ce sont des \textit{p-graphes} avec $p=1$.
 \item[Densité] La densité d'un graphe orienté est égal au cardinal de $U$ divisé par $N^2$ : $d=\frac{|U|}{N^2}=M$
 \item[Multi-graphes] Ce sont des \textit{p-graphes} sans les \enquote{flèches}. Ici, on n'a pas la notion de sens entre les noeuds liés. On a alors le modèle suivant : $G=(X, E)$ où $E$ représente les arêtes.\\Ici, la densité est égale à $d=\frac{M}{\frac{N(N-1)}{2}}$.
\end{description}

\section{Parties de graphes}
Un sous-graphe de G c'est la partie engendrée par A soit, pour un graphe $G=(X,U)$ où $A=X$, $SG_A(G)=(A, A^2\bigcap U)$.\\
Un graphe partiel de G est un sous graphe ayant le même nombre de noeuds mais on restreint le nombre d'arc. Soit, pour notre graphe $G=(X,U)$, on a $V\subset U$. Ce qui nous donne $SG_V(G)=(X,V)$.

\section{Parcours et connexité}
On a la notion de chemin. C'est une séquence de transition entre des noeuds. Un arc est un chemin de valeur 1.
Un chemin fermé est un circuit, un cycle ($u_1, u_2, u_3, \ldots, u_1$).\\
Un parcours élémentaire est un cycle, un chemin, ou une chaîne ne passant au plus qu'une fois par un sommet donné.\\
\subsection{Fermeture transitive d'un graphe}
On a $G=(X,\Gamma)$. $\Gamma : X \rightarrow P(x)$ soit $x \rightarrow {y|(x,y)\in U}.$\\
La fermeture transitive de x c'est l'ensemble des points y que je peux joindre en fonction d'une longueur donnée : $lim({x} \bigcup \Gamma(x) \bigcup \Gamma(\Gamma(x)))$.
\subsection{Connexité}
Un graphe est dit connexe ssi $\forall (x,y) \in X^2$ il existe un chemin allant de $x$ à $y$. $\forall y, y \in \Gamma(x)$.\\
GRAPHE 3\\
\subsection{Point d'articulation (Cut point, articulation point)}
C'est un sommet qui augmente le nombre de composante connexes si on l'enlève.
\subsection{Un isthme}
C'est un arc qui augmente le nombre de composantes connexes si on l'enlève.
\subsection{Un graphe fortement connexe}
Pour un graphe orienté $G(X,U)$ est dit fortement connexe ssi $\forall x,y; (x,y)\in U et (y,x)\in U$.

\section{Parcours eulériens et hamiltoniens}
Ces problèmes ont été rendus célèbres par les problèmes de tourmé ou du voyageur de commerce, etc\ldots\\
Un graphe dual c'est un graphe dans lequel on échange les arcs et les noeuds.
\subsection{Parcours eulériens}
Un parcours dans un graphe est eulérien ssi il passe une et une seule fois par toutes les arêtes de G.\\
GRAPHE\\
Ce graphe est un parcours eulérien. En effet, il est possible de passer par les noeuds une seule fois.\\
\begin{theo}
Théorème d'Euler\\
Un multigraphe admet un parcours eulérien si et seulement si :
\begin{itemize}
 \item il est connexe
 \item il possède 0 ou 2 sommets de degré impair.
 \begin{itemize}
  \item S'il a 0 sommet de degré impair, le parcours est un cycle
  \item S'il a 2 sommets de degré impair, le parcours est obligatoirement une chaîne reliant ces deux points.
 \end{itemize}
\end{itemize}
\end{theo}
\subsection{Parcours hamiltonien}
Un parcours $\mu$ est hamiltonien s'il passe une et une seule fois par tous les sommets de G.\\
Ce parcours est connu grâce au problème du voyageur de commerce (PVC)\footnote{Travelling Sales Man (TSM)}.

\section{Graphes particuliers}
\subsection{Graphes symétriques}
Si $(x,y) \in U \Leftrightarrow (y,x) \in U$
\subsection{Graphes antisymétriques}

\subsection{Graphes complets}
$\forall x \in X, \forall y \in X, (x,y) \in U$\\~\\
\begin{theo}
La clique\\
 La clique d'un graphe G est tout sous-graphe complet.
\end{theo}

\subsection{Graphes bipartites / multipartites}
$G=(X_1,X_2,\mu)$\\$X_1\bigcup X_2=X$\\$\forall (x,y) \in U, X \in X_1 et Y \in X_2$\\
GRAPHE\\
Dans le cas des graphes bipartites $K_{n_1,n2}$. Un graphe $K_{2,3}$ est un graphe dont les partites ont des point qui sont touts liés.

\subsection{Graphes planaires}
G est planaire ssi on peut le dessiner sans croisement d'arêtes.\\
Tous les graphes tels que $|X|<5$ sont planaires.\\
\begin{exo}
 Montrer que $K_{2,3}$ est planaire.
\end{exo}
CORRECTION = GRAPH\\
$K_5$ est le plus petit graphe complet non planaire.\\
$K_{3,3}$ est le plus petit graphe complet k-partite non planaire.
\begin{theo}
 Théorème de Kuratowski\\
 $G=(X,E)$ et planaire\\
 Ssi il ne contient aucun sous-graphe réductibles à $K_5$ ou $K_{3,3}$.\\
 Un sous-graphe $G_1$ est réductible à un sous-graphe $G_2$ si on peut rendre $G_1$ égal à $G_2$ en réduisant les noeuds et en contractant chaque sommet de degré 2.
\end{theo}
\subsection{Les arbres}
Ce sont des graphes connexe et sans cycle. Cela implique que $M=N-1$.\\
GRAPHE\\
Pour un arbre, il suffit d'enlever une arête pour augmente le nombre de composantes connexes.\\
Une arborescence, c'est un arbre orienté.\\
GRAPHE\\
\subsubsection{Les anti-arborescences}
Une anti-arborescence est un arbre dans lequel le sens des arcs a été inversé.

\subsubsection{Les arbres recouvrants}
On a : $G=(X,E)$ et $T$, un graphe partiel connexe de G.\\
$T$ est un arbre connexe recouvrant\footnote{Spanning tree} de $G$. On retrouve ces problèmes dans les arbres généalogiques, dans les hiérarchies, dans les réseaux, etc\ldots
