\rhead{13/10/2015}
\section{Fonction booléennes}
C'est :
\begin{equation}
 \left\{\left\{0;1\right\} ^{2}, \wedge, \vee, -\right\} \\
  Def :  \mathbb{F}_2^n \rightarrow \mathbb{F}_2
\end{equation}
On a ici les opérateurs \enquote{ou}, \enquote{et} et la complémentation.\\
On a ici un corps fini\footnote{Fined field}.\\
Table de vérité c'est un vecteur.\\
On a $\vee_{\alpha\in\mathbb{F}_2^n}x^\alpha$.\\
si $\alpha=(\alpha_1, \alpha_2, \ldots, \alpha_n)$, alors $x^\alpha=x_1^{\alpha}.x_2^{\alpha2}.x_3^{\alpha3}.\ldots.x_n^{\alpha n}$\\
CFN : Conjonctive normal form.\\
$\wedge(\vee_{\alpha i}~x_i^{\alpha i})$\\
$ (x_1\vee \overline{x_2} \vee x_3)(x_2 \vee \overline{x_1} \vee \overline{x_3}) $\\
Problème 3 sat\footnote{satisfaisant}. Résolvable par les sat solveur.\\
Une formule conjonctuve n'est pas normale. Les formes dijonctives normales ne sont pas unique. Ce phénomène est génant.\\
On va alors utiliser la forme algébrique normale.
\subsection{Forme algébrique normale}
Cela s'écrit de la manière suivante : $\bigoplus_{\alpha\in\mathbb{F}_2^n}~x^\alpha$\\
On a alors : $x_1.x_2\oplus x_1.x_3\oplus x_2.x_3$\\
Dans ce cas, $x_i^{\alpha i}=x_1$ si $\alpha i =1$ ou $x_i^{\alpha i}=1$ ou $x_i=0$\\~\\

$f: \mathbb{F_2^n}\rightarrow\mathbb{F_2}$\\
Le poids de Hamming de f est noté de la manière suivante :
\begin{equation}
 \omega t(f)=\left\|\left\{x\in\mathbb{F_2^n},  \right\}\right\|
\end{equation}

???\\
Soit f une fonction associant .... On appelle transformée de Walsh Fourier de f, la fonction \^{f} la fonction allant de $\mathbb{F}_2^n$ à $\mathbb{C}$ définie par :
\begin{equation}
 \hat{f}(u) = \sum_{}^{} (-1)^{f(x)}.(-1)^{<x.u>}
\end{equation}
Leame : On a $\hat{f}=2^n.2.ham(f, <x,u>)$\\
On a vu que $\hat{f}(0)=4$\\
$ham(x,y)=\omega t(x\oplus y$\\
Définition : Toutes les fonctions $\mathbb{F}_2$ linéaires sur $\mathbb{F}_2^n$ sont de la forme $<x,u>, \forall u$, soit :
\begin{equation}
 E x_i.u_i
\end{equation}
\begin{theo}
 \begin{equation}
  f(x)=2^n.(-1)^{f(x)}, \forall x
 \end{equation}
On a une quasi involutivité
\end{theo}
\begin{exo}
 Toute fonction $\mathbb{F}_2$ linéaire non constante est équilibrée.
\end{exo}
\begin{theo}
 Théorème de Parceval.\\
 Soit $f: \mathbb{F_2^n}\rightarrow\mathbb{F_2}$. Alors, 
 \begin{equation}
  \sum_{u\in\mathbb{F}_2^n} (\hat{f}(u))^2 = 2^{2n}
 \end{equation}
\end{theo}
Calcul tensoriel = matrices de matrices.\\
Calcul de la transformée de WF :\\
Méthode naïve : $f(u)=\sum_{} (-1)^{f(x)-2}$\\
\subsection{Transformée de WF rapide : algorithme de Cooley-Tukey}
Spectre de fourier pour la fonction tel que $...$. On calcule la totalité des valeurs du spectre de Fourier en on complexité logarithmique en temps et exponentielle en mémoire.\\
\begin{equation}
 Spectre_{WF}(f)=(0;4;4;0;4;0;0;-4)
\end{equation}
Application :\\
$f$ équilibrée : $\hat{f}(000)=0$\\
\begin{equation}
 P(f(u)=<x,u>)=\frac{1}{2}.\left(1-\frac{\hat{f}(u)}{2^n}\right)
\end{equation}
\begin{equation}
 P(f(u)=<x,u>)=\frac{1}{2}.\left(1-\frac{4}{8}\right)
\end{equation}
\begin{equation}
 P(f(u)=<x,u>)=\frac{1}{2}.\left(\frac{3}{2}\right)
\end{equation}
\begin{equation}
 P(f(u)=<x,u>)=\frac{3}{4}
\end{equation}~\\~\\
\begin{exo}
 Démonstration d'un corollaire :
 \begin{equation}
  HAM(f,<x,u>)+HAM(f,<x,u>\oplus1)=2^n
 \end{equation}
\end{exo}
