\section{Vision algébrique de la théorie des graphes}
\subsection{Matrices d'adjacences}
On a $G=\left(X,E\right)$ un graphe simple avec $|X|=n$.\\
La matrice d'adjacence de $G$ est un matrice de taille $n$. On a $A=[a_{ij}]_{i,j \leqslant n}$.
\begin{equation}
 a_{ij}=\left\{
    \begin{array}{ll}
     1~si~(i,j)\in E\\
     0~sinon
    \end{array}
  \right.
\end{equation}
\subsection{Spectre de graphes}
Spectre de graphe $G=(X,E)$ est noté de la matière suivante :
\begin{equation}
 Spec~G=
 \begin{pmatrix}
  \lambda_{0} & \lambda_{1} & \ldots & \lambda_{s-1}\\
  m(\lambda_{0}) & m(\lambda_{1}) & \ldots & m(\lambda_{s-1})
 \end{pmatrix}
\end{equation}
\subsection{Polynôme caractéristique}
...\\
Propriété : 
\begin{itemize}
 \item $C_1=0$
 \item $-C_2=$ nombre d'arêtes de G
 \item $-C_3=$ nombre de triangles dans G $\Leftrightarrow K_3$
\end{itemize}

Proposition : Le nombre de chaînes/cycles de taille $l$ dans $G$ joignant $x_i$ et $x_j$ est la valeur des $a_{ij}$ de $A^l$.\\
Proposition : Soit $G$ un graphe connexe de matrice d'adjacence $A$ et de diamètre $d$. Alors, la dimension de $A(G)$ est au moins égale à $d+1$.\\
Corollaire : Soit un graphe $G$ connexe tel que $|x|=n$ et de diamètre $d$. On a $v$ a valeur propre avec $d+1 \leqslant v\leqslant n$