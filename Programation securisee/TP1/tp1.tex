\documentclass[a4paper,10pt]{article}
\usepackage[utf8]{inputenc}
\usepackage{tabularx}
\usepackage[french]{babel}
\usepackage{csquotes}
\usepackage[none]{hyphenat}
\usepackage{graphicx}
\usepackage{hyperref}
\usepackage{fancyhdr}
\usepackage{float}
\usepackage{caption}
\usepackage{listings}
\usepackage{spverbatim}

\lhead{Antoine Puissant}
\chead{Progammation (in)sécurisée - TP1}
\rhead{2015-2016}

\pagestyle{fancy}

\sloppy

\begin{document}

%\maketitle
\begin{titlepage}
      \begin{center}   
        \Huge
        \textbf{Rédaction du mémoire de fin d'études}
        
        %\vspace{0.5cm}
        \LARGE
        ~
        
%         \vspace{0.5cm}
        
        \vfill
        \begin{figure}[H]
	    \centering
	    \begin{minipage}{0.9\textwidth}
		\centering
		\includegraphics[width=\textwidth]{./img/esiea.jpeg}
	    \end{minipage}\hfill
	\end{figure}
        \vfill
        
        \vspace{0.5cm}
        
        Méthodologie
        
        \vspace{2cm}
        \textbf{Antoine Puissant}\\
        \vspace{0.8cm}
        \Large
        \underline{Enseignant :} Mme. Klinger\\
        \vspace{0.5cm}
        2014 - 2015%\today
        
    \end{center}
\end{titlepage}

\begin{abstract}
L'objectif de ce TP est d'analyser un code C/C++ afin de trouver une vulnérabilité dans ce dernier. Une fois trouvée, nous devrons exploiter la vulnérabilité en implémentant un \textit{stack based buffer overflow}.
\end{abstract}

\newpage

\tableofcontents

\rhead{04/11/2015}
\section{titre}
La norme 802.1X (c.f. \cite{8021X}) permet de ne pas accéder au réseau si l'utilisateur ne s'est pas authentifié (même câblé).\\
Le contrôle et le pentest font partie de l'audit.\\
\begin{description}
 \item[Pentest] Pour  un pentest, on se place en mode boite noire : on a la place de l'attaquant, on ne connaît pas l'architecture interne d'un réseau. De manière générale, un pentest est assez ciblé. Il n'y a souvent que peu de machines cibles.
 \item[Audit] L'audit se place à l'intérieur d'un réseau, on prend la position du défenseur. C'est un étude qui est très grande. On va, par exemple, évaluer le réseau entier de l'entreprise. On va pouvoir faire du pentest sur quelques machines (les machines critiques) au sein du réseau.\\Le référentiel de l'audit est l'état de l'art.\\Ainsi, le résultat d'un audit est composé de préconisation.
 \item[Contrôle] Pour un contrôle, on se positionne dans une entreprise mûre en sécurité. Le RSSI a alors une politique de sécurité. Ainsi, pour un contrôle, on va vérifier que la politique de sécurité est bien appliquée.\\Le résultat est composé d'ordres à faire appliquer aux salariés et de conseils d'améliorations de la politique de sécurité du RSSI.
\end{description}
Les normes ISO27000X ($0<X<40$) (c.f. \cite{ISO27}) vont permettre de réaliser une politique de sécurité en fonction de normes prédéfinies.\\
Pour les entreprises de la finance (banques, bourses, etc\ldots), il existe une norme obligatoire en Europe : PCI-DSS.\\
Les OIV\footnote{Organisme d'Importance Vitale} doit se faire auditer la maîtrise de la continuité : des exercices sont fait tous les ans afin de vérifier comment l'entreprise réagit en cas d'interruption, ce qu'elle fait pour redémarrer sa production.\\
Lors d'un audit, tout le technique est pris en compte. Cela comprend son environnement physique, technique tout en se basant sur la référence.
\subsection{L'audit}
Il se compose de trois phases.
\subsubsection{La prise de contact, phase préparatoire}
Le client fait une demande d'audit. Pour cela, on passe par les commerciaux afin de réaliser les négociations.\\
On réalise une visite préparatoire. On va alors chez le commanditaire. Le commanditaire n'est pas obligatoirement le client, cela peut-être une entité supérieure (ANSSI pour les OIV).\\
On va ensuite chez le client pour se présenter. On va alors convier à cette journée de réunion le RSSI, les administrateurs réseau (ou l'entreprise qui gère l’infogérance). Cette réunion préparatoire va permettre de définir le périmètre mutuel d'audit.\\
On va ensuite signer la charte de confidentialité des données de l'entreprise (NDA). On va alors assurer de la manière dont les données seront sauvegardées, effacées, détruites, etc\ldots\\
On signe ensuite une charte d'audit, un protocole et un contrat.\\~\\
Une fois cette réunion finie, il peut passer un peu de temps. Durant cette période on va planifier les interviews que l'on va réaliser avec le personnel de client.
\subsubsection{L'audit}
On va alors faire de l'investigation organisationnelle, vérifier comment l'organisation de l'entreprise est faite. On investigue aussi l'informationnelle. Cela permet de connaître l'e-réputation de l'entreprise. On va trouver cela en sources ouvertes.\\
On va pouvoir, dans le cas où il y a plusieurs auditeurs, faire des comptes-rendus chaque soirs pour se tenir au courant.\\
En fin d'audit, il est courant de faire un compte-rendu à chaud.\\
On peut aussi faire une séance de sensibilisation du personnel.
\subsubsection{Le compte-rendu, le livrable}
Il faut prendre le temps pour pouvoir ordonner toutes les données récupérer, tout analyser. Il faut se laisser le temps de l'analyse afin de pouvoir analyser correctement et réaliser les bonnes recommandations.

\section{Rendez-vous de présentation}
\begin{enumerate}[I]
 \item Présentation des auditeurs, de qui on est.
 \item Méthodologie. C'est ce qu'on compte faire, comment.
 \item Commanditaire et ses valeurs. On défini avec eux ce qu'il faut sécuriser, ce qui est primordial.
 \item Périmètre de l'audit. Ça peut-être les machines de l'entreprise, les accès sur des serveurs externes.
 \item Modalités.
\end{enumerate}

\section{Retour d'expérience -- RETEX}
\begin{itemize}
 \item Pas d'analyse de risque
 \item Pas de cloisonnement des données
 \item Pas de notion de secret, beaucoup de gens donnent trop d'informationnelle
 \item Pas de marquage sur les documents (niveau de confidentialité)
 \item Pas d'étude des pannes, de snapshots, de backup, etc\ldots
 \item Pas d'organisation du travail
 \item Pas de délimitation de l'inclusion des tiers (SSII)
 \item Personnel non sensibilisé
 \item Problème d'employé malveillant
 \item Il faut interdire de pouvoir s'authentifier à deux endroits différents en simultané
\end{itemize}

\section{Droits de devoirs des administrateurs}
Il est interdit de lire les messages comprenant la notion de \enquote{personnel} ou avec des noms et prénoms.\\
Il est possible d'accéder aux fichiers professionnels. L'accès fortuit aux données personnelles (scan de dossiers, etc\ldots), cela n'est pas puni si l'accès au documents n'est pas fait.Il ne faut surtout pas faire fuiter l'information personnelle sur laquelle on est tombé.\\
Dans le cas d'un audit, il est autoriser d'investiguer et de tomber sur des documents personnels. On est mandaté pour cela. Il faut rester discret. La discrétion est le maître mot en cas de découverte de documents particuliers.\\
En cas de problème détecté, un administrateur peut accéder aux documents professionnels et personnel pour analyse. Il n'a cependant pas le droit de modifier les contenus. Si le problème est lié à la sécurité, on peut supprimer la donnée virolée dans le cadre de la politique de sécurité.\\
Si aucun problème n'est rencontré, aucune investigation n'est autorisée.\\
En cas d'intervention sur un poste, on essaye d'être avec la personne. On n'ouvre pas les données personnelles et on peut détruire les données personnelles avec son accord.
\section{Protocol SMB}
SMB vient de IBM NETBIOS et à été repris et modifié par Microsoft.\\
Il est dédié uniquement aux réseaux LAN car les paquets font environ 50 octets. Cela va permettre d'être super réactif sur les LAN. Sur un WAN ça ne va pas fonctionner d'une bonne manière car les petits paquets vont surcharger les routeurs. Les prestataires (opérateurs) vont alors bloquer le protocole.\\
Le projet de portage de SMB sur Linux s'appelle SAMBA.\\
Le nom de SMB est aussi cifs.
\newpage
\section{Infrastructure du réseau Internet}
Cloud computing :
\begin{itemize}
 \item IaaS = Achat de machines physique. Plus cher mais plus libre.
 \item PaaS = Achat d'un VM. On peut mettre l'OS que l'on veut mais on n'a pas l'infirmation du serveur.
 \item SaaS = Achat d'un service uniquement.
\end{itemize}
SDNL = débit montant = débit descendant\\
FTTH/FTTB = fibre.\\
Opérateurs satellitaires (hors de la législation Française)\\
Comment sécuriser un réseau d'entreprise comprenant un réseau non classifié (Internet) et un réseau confidentiel ?
\begin{enumerate}
 \item Les 2 réseaux sont séparés physiquement. Même pas de VLAN. Les deux réseaux n'ont aucuns lien.
 \item Pour le réseau NP, on va mettre une sécurité (firewall)
 \item Mise en place d'une station blanche pour transférer des données du NP au P.
 \item Mise en place d'un DHCP snooping pour sécuriser le DHCP. Il va permettre en configurant le switch, de lui dire sur quels ports doivent aller les réponse DHCP (sur le port du serveur DHCP). Il faut alors un switch intelligent pour qu'il puisse aller sur la couche IP.
 \item Chiffrer les Wi-Fi avec des clés primaires simples.
 \item Limiter la puissance du Wi-Fi pour ne pas rayonner la où on n'en à pas besoin.
 \item Cacher le SSID pour le réseau d'entreprise (visible pour le guest)
\end{enumerate}
Vulnerabilités du DHCP : 
\begin{itemize}
 \item DDoS : on fait des requêtes jusqu'à que le DHCP n'aura plus d'adresses disponibles.
 \item Faire un serveur DHCP autre qui répondra plus rapidement que le vrai. Ainsi, les machines vont passer par le faux serveur DHCP.
\end{itemize}
\subsection{Obligations légales}
Toute personne physique ou morale qui propose un accès aux services de communication ou au stockage, au public, doivent conserver toutes les infos permettant d'identifier toutes les infos (dates, durée, identité; \ldots) des utilisateurs.\\
La CNIL dit qu'un utilisateur est composé d'un nom et un prénom.\\
Il faut mettre en place un NAC\footnote{\textit{Network Access Derivation}}. On le place en coupure (pas en dérivation, si lui tombe, tout le réseau tombe pour pas qu'il loupe des informations).\\
Trois solutions : 
\begin{itemize}
 \item Louer un produit qui fait cela (il faut que ça soit écrit sur le contrat)
 \item Acheter une Appliance\footnote{OS modifié pour une application particulière puis monté} (Utopia, StormShield, DSCbox, etc\ldots)\\Il faut contrôler la solution pour être sur que cela fonctionne correctement !
 \item Réaliser la solution en interne.\\Il faut aussi contrôler ! Soit par le RSSI soit par un audit externe.
\end{itemize}
Pour le contrôle, 
\begin{itemize}
  \item Il faut tout loguer, tous les ports ! Que ça soit de l'IRC, de la VoIP, du HTTP, etc\ldots
\end{itemize}
Autres NAC :
\begin{itemize}
 \item IPCOP
 \item PFSense
\end{itemize}
Le problème des NAC \enquote{génériques} sont qu(ils vont logguer de manière simpliste. Il faut, pour la loi française pouvoir associer une personne à un moment donné.
\newpage
\listoffigures\addcontentsline{toc}{section}{\listfigurename}
\end{document}
