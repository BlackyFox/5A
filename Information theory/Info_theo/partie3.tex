\chead{Information theory - 08/09/2015}
\section{Communication through noisy channels}
Peut-on lutter contre les altération de la nature et si oui, comment ?\\
Un canal de communication est une boîte noire. Cela peut être un espace mémoire (HDD, RAM) sur laquelle on va lire et écrire.\\
Les canaux sans mémoire (un symbole ne dépend pas de ceux qui le précèdent) discrets (un nombre fini).\\
Un canal est défini par une matrice de canal/distribution.\\
La proba $p$ est la proba d'erreur.\\ 
\subsection{Codage}
La probabilité que le symbole binaire soit altéré est de $p$.\\
La probabilité qu'un symbole ne soit pas changé est $q=1-p$.\\
Sur $n$ mots émis, la probabilité que les mots soient émis correctement est de :\\
\begin{center}
  $((1-p)^3)^n$\\ 
\end{center}
soit,\\
\begin{center}
  $(1-p)^{3n}$\\
\end{center}
Doubler le message ne fait que de le détection d'erreurs.\\
Pour de la correction d'erreurs, on va pouvoir faire de la \enquote{k-répétition} :
\begin{center}
  1 0 1 1 
\end{center}
est le message de base. Celui envoyé est le suivant :
\begin{center}
  111 000 111 111
\end{center}
On reçoit alors :
\begin{center}
  101 110 111 101
\end{center}
On va alors faire un \enquote{vote majoritaire} et avoir le message suivant :
\begin{center}
  1 1 1 1
\end{center}
On a donc ici un canal qui bruite plus que prévu. On va donc faire une répétition de 5 :
\begin{center}
  11111 00000 11111 11111
\end{center}
Et on reçoit :
\begin{center}
  11011 01100 11111 11011
\end{center}
Le vote majoritaire donne alors le résultat suivant :
\begin{center}
  1 0 1 1 
\end{center}
\subsection{Décodage}
Une règle de décodage doit permettre de produire une partition.\\
L'objectif est de minimiser le taux d'erreur résiduelle.\\
L'observateur ne connaît pas des probabilité de distribution de ce qui lui arrive.\\
Lorsque les mots de codes ont la même probabilité, les deux règles sont équivalentes.\\
\section{Capacité}
\section{Second théorème de Shannon}
\section{Conclusion}
Moyennant un mécanisme de redondance, on pourra toujours s'assurer que Bob et Alice aient la même information.\\
On est ici dans la sûreté, la nature n'est pas mal	icieuse (contrairement à l'attaquant qui va changer en permanence $p$ de bruitage).